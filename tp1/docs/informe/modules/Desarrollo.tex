\section{Desarrollo}

\subsubsection{Consideraciones preliminares} %TODO: mejor nombre

Para ejecutar el algoritmo de \emph{Page Rank} debemos resolver el sistema de ecuaciones lineales $A x = x$, donde la matriz A $\in \mathbb{R}^{nxn}$ esta definida como:

\begin{equation}
    a_{ij} = \left\{
            \begin{array}{ll}
                 (1-p)/n + (pw_{ij})/c_j      & \mathrm{si\ } c_j \neq 0 \\
                 1/n & \mathrm{si\ } c_j = 0
            \end{array}
        \right.
\end{equation}

Queremos demostrar que $A = pWD + e.z^t$, donde:  
\begin{itemize}
	\item p $\in \mathbb{R}$ la probabilidad de seguir un link de la página actual.  
	\item W $\in \mathbb{R}^{nxn}$ es la matriz de conectividad.  
	\item  e $\in \mathbb{R}^{n}$ es un vector de unos.
    \item D $\in \mathbb{R}^{nxn}$ es una matriz diagonal con cada elemento de la misma de la forma:     
    \begin{equation}
    	\label{defd}
        d_{jj} = \left\{
                \begin{array}{ll}
                     1/cj & \mathrm{si\ } c_j \neq 0 \\
                     0    & \mathrm{si\ } c_j = 0
                \end{array}
            \right.
    \end{equation}
    \item z $\in \mathbb{R}^{n}$ es un vector de la forma:  
    \begin{equation}
    	\label{defz}
        z_{j} = \left\{
                \begin{array}{ll}
                     (1-p)/n      & \mathrm{si\ } c_j \neq 0 \\
                     1/n & \mathrm{si\ } c_j = 0
                \end{array}
            \right.
    \end{equation}
\end{itemize}

La matriz $B = p.W$, por ser multiplicacion de un escalar con una matriz, es de la forma:

\begin{equation}
	\label{defb}
	b_{ij} = p.w_{ij}
\end{equation}

Luego sea la matriz $R = BD = pWD$. Entonces cada elemento de R queda definido como $R_{ij} = \sum_{k=1}^n b_{ik} d_{kj}$

Pero, como la matriz D es nula excepto en la diagonal, se cancelan todos los términos de la sumatoria excepto cuando k = j, luego: 
\begin{equation}
R_{ij} = b_{ij} d_{jj}
\end{equation}

Reemplazando por las definiciones \ref{defb} y \ref{defd}: 

\begin{equation}
	\label{defr}
    R_{ij} = \left\{
            \begin{array}{ll}
                 (p.w_{ij})/c_j & \mathrm{si\ } c_j \neq 0 \\
                 0              & \mathrm{si\ } c_j = 0
            \end{array}
        \right.
\end{equation}

Ahora, si tomamos $e.z^t$ y la nombramos Q, entonces Q $\in \mathbb{R}^{nxn}$ (porque $e \in \mathbb{R}^{nx1}$ y $z^t \in \mathbb{R}^{1xn}$). Luego $Q_{ij} = \sum{k=1}^1e_i.z_j$.

Observando que cada fila de $e$ tiene un solo elemento podemos escribir:  
$Q_{ij} = e_i.z_j$

Siendo $\forall i, e_i = 1$ y los elementos $z_j$ definidos como en la ecuacion \ref{defz}:

\begin{equation}
	\label{defq}
    Q_{ij} = \left\{
            \begin{array}{ll}
                 (1-p)/n      & \mathrm{si\ } c_j \neq 0 \\
                 1/n & \mathrm{si\ } c_j = 0
            \end{array}
        \right.
\end{equation}

Entonces $pWD + e.z^t = R + Q$. La suma esta bien definida al ser R $\in \mathbb{R}^{nxn}$ y Q $\in \mathbb{R}^{nxn}$. A esta matriz la llamo C y quiero ver que $C=A$:

$C_{ij} = R_{ij} + Q_{ij}$

Reemplazo por las definiciones \eqref{defr} y \eqref{defq}:

\begin{equation}
	\label{defc}
    C_{ij} = \left\{
            \begin{array}{ll}
                 (1-p)/n + (p.w_{ij})/c_j & \mathrm{si\ } c_j \neq 0 \\
                 1/n + 0                  & \mathrm{si\ } c_j = 0
            \end{array}
        \right.
\end{equation}

Luego nos queda $C=A$.
